% Template for PLoS
% Version 1.0 January 2009
%
% To compile to pdf, run:
% latex plos.template
% bibtex plos.template
% latex plos.template
% latex plos.template
% dvipdf plos.template

\documentclass[10pt]{article}

% amsmath package, useful for mathematical formulas
\usepackage{amsmath}
% amssymb package, useful for mathematical symbols
\usepackage{amssymb}

% graphicx package, useful for including eps and pdf graphics
% include graphics with the command \includegraphics
\usepackage{graphicx}

% cite package, to clean up citations in the main text. Do not remove.
\usepackage{cite}

\usepackage{color} 

% Use doublespacing - comment out for single spacing
%\usepackage{setspace} 
%\doublespacing


% Text layout
\topmargin 0.0cm
\oddsidemargin 0.5cm
\evensidemargin 0.5cm
\textwidth 16cm 
\textheight 21cm

% Bold the 'Figure #' in the caption and separate it with a period
% Captions will be left justified
\usepackage[labelfont=bf,labelsep=period,justification=raggedright]{caption}

% Use the PLoS provided bibtex style
\bibliographystyle{plos2009}

% Remove brackets from numbering in List of References
\makeatletter
\renewcommand{\@biblabel}[1]{\quad#1.}
\makeatother


% Leave date blank
\date{}

\pagestyle{myheadings}
%% ** EDIT HERE **


%% ** EDIT HERE **
%% PLEASE INCLUDE ALL MACROS BELOW

%% END MACROS SECTION

\begin{document}

% Title must be 150 characters or less
\begin{flushleft}
{\Large
\textbf{Title}
}
% Insert Author names, affiliations and corresponding author email.
\\
Author1$^{1}$, 
Author2$^{2}$, 
Author3$^{3,\ast}$
\\
\bf{1} Author1 Dept/Program/Center, Institution Name, City, State, Country
\\
\bf{2} Author2 Dept/Program/Center, Institution Name, City, State, Country
\\
\bf{3} Author3 Dept/Program/Center, Institution Name, City, State, Country
\\
$\ast$ E-mail: Corresponding author@institute.edu
\end{flushleft}

% Please keep the abstract between 250 and 300 words
\section*{Abstract}

% Please keep the Author Summary between 150 and 200 words
% Use first person. PLoS ONE authors please skip this step. 
% Author Summary not valid for PLoS ONE submissions.   
\section*{Author Summary}

\section*{Introduction}
Evolution by natural selection has led to a hierarchical relationship among all living organisms.  Thus, species are categorized using a taxonomic hierarchy, starting with the binomial species name (e.g, \emph{Homo sapiens}), moving up to genus (\emph{Homo}), then family (\emph{Hominidae}), and on up to Domain (\emph{Eukarya}). Biologists, whether studying organisms at the cell, organismal, or community level, can put their study taxa into taxonomic context, allowing them to know close and distant relatives, find relevant literature, and more. Discovering the correct taxonomic names is, unfortunately, not straightforward. Taxonomic names often change due to name changes at the generic or specific levels, lumping or splitting lower taxa (genera, species) among higher taxa (families), and name spelling changes. In addition, there is no one authoritative taxonomic names source. Instead, there are essentially competing sources (e.g., uBio, Tropicos, ITIS) that may have different accepted names for the same taxon. The goal of taxize, an R package in development, is to make all use cases having to do with retrieving and resolving taxonomic names easy and replicable. 

Taxonomic data is getting easier to obtain through web interfaces (e.g., \cite{eol}). However, there are a number of good reasons to obtain taxonomic information programatically rather than through a web interface. First, if you have more than a few names to lookup on a website, it can take quite a long time to enter each name, get data, and repeat for each species. Second, programatically getting taxonomic names solves the first problem by looping over a list of names. In addition, doing taxonomic searching, etc. is reproducible. With increasing reports of irreproducibility in science \cite{stodden2010,zimmer2012}, it is extremely important to make science workflows repeatable. Science workflows can now easily incorporate text, code, and images in a single executable document \cite{yihui2013}. 

The R language is the dominant language used by biologists (reference), and now has over 5,000 packages on the R package repository (CRAN) and more than 2,500 packages on other repositories to extend R. R is great for manipulating, visualizing and fitting statistical models to data. However, the key missing piece in R is the ability to get data from the internet within R. Getting data from the web will be increasingly common as more and more data gets moved to the cloud. Increasingly, data is available from the web via API's, or application programming interfaces. These are bits of code that allow computers to talk to one another using code that is not human readable, but is machine readable. Web APIs often define a number of methods that allow users to search for a species name, or retrieve the synonyms for a species name, for example. A further strength of APIs is that they are language agnostic, meaning that data can be consumed in almost any computing context, allowing users to interact with the web API without having to know the details of the code. Whereas, if data are stored in an Excel file, for example, the file can only be opened in a few programs. 

In taxize, we have written a suite of R functions that interact with many taxonomic data sources via their web APIs (Table~\ref{tab:a}). The interface to each function is usually a simple list of species names, just as a user would do with a web API. Therefore, we hope moving from a web to R interface for taxonomic names will be relatively seamless (if one is already nominally familiar with R). 

Here, we justify the need for taxize, discuss our data sources, and run through a suite of use cases to demonstrate the variety of ways that users can interact with taxize. 

% Results and Discussion can be combined.
\section*{Results}

\subsection*{Subsection 1}

\subsection*{Subsection 2}

\section*{Discussion}

% You may title this section "Methods" or "Models". 
% "Models" is not a valid title for PLoS ONE authors. However, PLoS ONE
% authors may use "Analysis" 
\section*{Materials and Methods}

% Do NOT remove this, even if you are not including acknowledgments
\section*{Acknowledgments}


%\section*{References}
% The bibtex filename
\bibliography{template}

\section*{Figure Legends}
%\begin{figure}[!ht]
%\begin{center}
%%\includegraphics[width=4in]{figure_name.2.eps}
%\end{center}
%\caption{
%{\bf Bold the first sentence.}  Rest of figure 2  caption.  Caption 
%should be left justified, as specified by the options to the caption 
%package.
%}
%\label{Figure_label}
%\end{figure}


\section*{Tables}
%\begin{table}[!ht]
%\caption{
%\bf{Table title}}
%\begin{tabular}{|c|c|c|}
%table information
%\end{tabular}
%\begin{flushleft}Table caption
%\end{flushleft}
%\label{tab:label}
% \end{table}

\end{document}

